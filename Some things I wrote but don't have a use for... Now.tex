
Once upon a time in London, there was a clever fellow named Isaac Newton. You might have heard of him. \\
He had a friend who was into trading and investments, and one day, while in the pub, this friend came to Isaac with a question. He said, 'Isaac, I've put 100 euros in a trading company, and they promise a 10\% interest rate every year. When will all come back double?'\\
The bartender replied, "Sir, its quite simple. Over ten years, 10\% increase will lead to 100\% increase and your money is doubled."\\
Isaac scratched his head for a moment, then he picked up a pulled out a piece of chalk and started scribbling on the bar table. He said, 'It's not that simple. See after the first year, your money will go up by 10\%, so it'll become 110 euros. But here is where things get a bit more complex: You don't just earn 10\% on the original 100; you also earn it on the extra 10 you made in the first year. So, that's 10\% of 110, which makes it 121.'\\
His friend raised an eyebrow and said, 'And what about the third year?'\\
Isaac grinned and replied, 'Well, it's the same idea. You take that 121, add 10\%, and voila, you've got 133.10 euors. It keeps going like this."\\
The bartender asked, "So it will double in less then 10 years. If I calculate correctly, in 9 years it would have become 214 euros."\\
Newton smiled, "I reckon your calculation must be accurate. After all you keep our bar tab in your head and yet don't forget anything substantial. But they are incorrect."\\
"What?" They both said.\\
"It's not like the 10 euors appeared only at the end of the year. They are appearing every month, every day, every hour as we speak. Which means that the money is increasing on an hourly basis."\\
"So when will it double?" The friend asked.\\
The bartender had pulled out his own chalk, "10\% in one year means about $0.001\%$ per hour. That will probably double in 8 years."\\
Newton smiled, "But it's growing every minute."\\
"With all due respect, sir, we can keep going like this forever."\\
"That's the point.", Newton put the chalk back in his pocket. "But the money will double in 7 years."\\
"How?" the bartneder asked. The friend was no longer intrested.\\
"Come with me, I'll tell you."\\

% Perdegogic remarks on Calc-1 chapter
You may have noticed that we didn't have many inline exercises for you to solve in this chapter. This was because unlike all other chapters in this book, this chapter was more about understanding the theory than using it. \\
This chapter was almost the entirety of Calculus I, including many of the proofs you don't usually get to see.  They were included 
While we have covered some tricks and techniques which are not part of the usual course, while have not covered theorems which are part of Calc-I. Those will be covered in the chapter which is equivalent to Calc-II. 

% Quadratic Residue
\section{Quadratic Residues}
We have already used them quite a bit, this is only a formality.\\
\begin{definition}
    $a$ is called a quadratic residue of $p$ if $a \equiv x^2 \pmod{p}$. Otherwise it is called a called a quadratic non-residue.
\end{definition}
Here we note that $0$ will and will not be a quadratic residue due to the fact that it doesn't have an inverse or negative value, some theorems will not consider it an residue while some will. I'll normally mention which case we are talking about\\
You also may have noticed that modulus $p$ tends to have $\frac{p-1}{2}$ quadratic residues, based on the ones we removed in chapters past.\\
This is true as $1^2$ and $(p-1)^2$ are congruent modulo $p$ as $(-1)^2=1^2$. This means that we actually don't need to solve for all values, only $1-\frac{p-1}{2}$. That's quite great, isn't it?\\
Now let's talk about some notation.\\
\begin{definition}
[Legendre Notation]
    \[(\frac{a}{p}) = \begin{cases}
    1, & \text{if } a \text{ is quadratic residue of } p\\
    0, & \text{if } a \text{ is divisible by } p\\ 
    -1, & \text{otherwise}\\ 
    \end{cases}\]
\end{definition}
This may seem like a strange definition but it is motivated by the fact that the product of two non-zero quadratic residues is a quadratic residue. The product of two quadratic non residues is also a quadratic residue. However, the product of a quadratic residue with a non-residue is a non residue.\\
We can prove them by letting $p\equiv x^2 \pmod{p}$, $q\equiv y^2 \pmod{p}$ and $r\equiv z \pmod{p}$ where $z$ is not a perfect square.\\
$pq \equiv (xy)^2 \pmod{p}$ which is a quadratic residue.\\
$pr\equiv x^2z \pmod{p}$ which cannot be a square as $z$ is square free.\\
The product of non roots uses a bit more machinery.\\
\begin{proof}
   Let $X = \{x_1, x_2, \dots , x_n\}$ be the set of quadratic residues for some prime $p$. Also let the set of non quadratic residues be $y = \{y_1, y_2, \dots , y_n\}$. It is obvious that $X \cup Y = \{1, 2, 3, \dots , p - 1\} = S$,\\
   As we have already studied, $nS \equiv S \pmod{p}$.  As the product of non-quadratic to quadratic residue is non-quadratic, $nX$ is not quadratic. Which means $nY$ are all quadratic roots which means the product of non-quadratic roots is quadratic.\\
\end{proof}



Putnam 1985, A2

Finally, if you attempt to read this without working through a significant number of exercises (see §0.0.1), I will come to your house and pummel you with [Gr-EGA] until you beg for mercy. It is important to not just have a vague sense of what is true, but to be able to actually get your hands dirty. As Mark Kisin has said, “You can wave your hands all you want, but it still won’t make you fly.”

— Ravi Vakil, The Rising Sea: Foundations of Algebraic Geometry


Binomial theorem states that upon expanding $(a+b+c \dots)^n$, we will get $a^p b^q c^r \dots$ such that $p+q+r + \dots =n$ in all possible ways. For $(a+b)^n$, we can find the coefficients rather easily.