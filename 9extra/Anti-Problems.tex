\chapter{Anti-problems}
One of the first chapters of this book contained questions which had complex solutions but easy to understand formulations. As I close this book, this chapter is the anti-thesis. The questions will have complex formulations but very simple solutions.\\
However, these are not birthday cakes. The solutions tend to be conceptual and hard to notice. Such problems were earlier called 'Jewish Problems' as Russian Anti-Semitic professors to keep out the Jews and other 'undesirables' from their institutes. While you will surly find some problems fell easy, that is mostly as we have explored such ideas before, in those days, such access was rare. In addition, the students were given these problems one after another until they failed one of them, at which point they were given a failing mark.\\
They were basically like "Olympiad-style" questions, except that you were expected to answer them with far less time than in an actual Olympiad, and work them out orally. Sure you could probably work a few out with paper and pencil given the time, but perhaps you wouldn't always be given a paper and pencil. Often the answer itself would be something trivial or even intuitive, but the trick was rigorously proving your answer beyond the shadow of a doubt. Basically you could be asked to stop at any time and explain the reasoning to your interviewer, and perhaps even asked to take a different line of reasoning. Also, more than the problems being unsolvable - they were also generally designed to be ambiguous in such a way that the examiner could pick whichever interpretation made the candidate's solution invalid. \\
The problem with the "Jewish questions" was not that they were difficult. It was that they were being administered unfairly.\\
It is not wrong for an university to ask these sorts of questions on their entrance exams, if they are looking to select for talented applicants who can come up with their own solutions and/or have already learned concepts more advanced than just high school common core.\\
The problem with the "Jewish questions" was that not all applicants would be asked them. Most applicants would get the fairly standard "solve this quadratic equation" type of questions. But when the examiner would see somebody from an "undesirable caste" of people (e.g. a Jewish student), they would swap the typical questions with one of these problems, which they would almost surely be unable to solve in the time given.\\
And the entire point of doing this was plausible deniability: The solutions to these problems, once you already know the right way to do it, when written out, do not take up any more space than a solution to a more standard problem would. Therefore, the examiner/institution could feasibly claim impartiality, since to an uninitiated inspector both types of questions seem "roughly similar in difficulty". The institution can pretend to an outside observer that all students are being tested "fairly", since the correct solutions to both kinds of problems look "similar"; while at the same time immediately offing any "undesirable" applicants using a method that is hard to prove (or even spot in the first place) if you are yourself not an expert in mathematical problem solving and testing.\\
But I will not bore you with history. Here are the problems:\\
\section{Anti-Problems}
\begin{xcb}{Exercise}
\begin{enumerate}
    \item 
\end{enumerate}
\end{xcb}