\chapter{Algebraic Manipulations}
You are already expected to know basic algebra, I think it will be embarrassing for both of us if we need to discuss $x-3=5 \iff x=8$\\ Hence, I hope that you not will wonder what $x,y$ or $z$ has to do with math.\\
We'll study about manipulation in this chapter. Manipulation normally is a bad thing, its the act of saying or doing things to control or influence (a person or situation) cleverly or unscrupulously, for your gain. We should not be manipulative and be away from such people.\\
However, when we refer to manipulation in math, we refer to playing around with an algebraic equation to make it more favorable or easy to solve. We'll study some methods of manipulation in this chapter: 
\section{Binomial Theorem}
If you have already studied algebraic identities, you may know that $(a+b)^2=a^2+2ab+b^2$ and $(a+b)^3=a^3+3a^2b+3ab^2+b^3$. These are normally derived by opening the brackets and multiplying(FOIL). But how do we find $(a+b)^4$ or worse $(a+b)^{10}$. \\
We can notice that all the terms of the expansion of $(a+b)^k$ are $a^mb^n$ where $m+n=k$. Does this give you a feel of PnC?
\begin{theorem}
    [Binomial Theorem]
    $(a+b)^n = \sum_{k=0}^{n}\binom{n}{k}a^{n-k}b^k$
\end{theorem}
\begin{proof}
    As expected, The Binomial Theorem has a nice combinatorial proof:
We can write $(a+b)^k=\underbrace{ (a+b)\cdot(a+b)\cdot(a+b)\cdot\cdots\cdot(a+b) }_{k}$. Repeatedly using the distributive property, we see that for a term $a^m b^{k-m}$, we must choose $m$ of the $k$ terms to contribute an $a$ to the term, and then each of the other $k-m$ terms of the product must contribute a $b$. Thus, the coefficient of $a^m b^{k-m}$ is the number of ways to choose $m$ objects from objects $k$, or $\binom{k}{m}$. Extending this to all possible values of $m$ from $0$ to $k$, we see that $(a+b)^k = \sum_{m=0}^{k}{\binom{k}{m}}\cdot a^m\cdot b^{k-m}$, as claimed.
\end{proof}
We also need to note that:\\
\begin{theorem}
[Binomial Approximation]
    $(1+x)^n=1+nx$ for $x << 1$
\end{theorem}
\begin{proof}
    $(1+x)^n\\
    = 1+\binom{n}{1}x+\binom{n}{2}x^2+\dots$\\
    As $x<<1$, then $x^2$ will be insignificant in comparison to $1$.\\
    $(1+x)^n=1+nx$
\end{proof}
Let's try a problem:\\
\begin{example}
    (JEE Adv 2023)Let $a$ and $b$ be two nonzero real numbers. If the coefficient of $x^5$ in the expansion of $(ax^2+\frac{70}{27bx})^4$ and the coefficient of $x^{-5}$ in $(ax-\frac{1}{bx^2})^7$ are equal, then what is the value of $2b$?
\end{example}
\begin{proof}
    [Solution]
    Coefficient of $x^5$ in $(ax^2+\frac{70}{27bx})^4$ will be $\binom{4}{k} a^k (\frac{70}{27b})^{4-k}$. We also know $(x^2)^k \cdot \frac{1}{x^k}=x^5$.\\
    Hence, we can figure $k=3$. Thus, the coefficient is $\binom{4}{3} a^3 (\frac{70}{27b})$.\\
    Lets now do the same for the coefficient of $x^{-5}$ in $(ax-\frac{1}{bx^2})^7$. The coefficient is: $\binom{7}{L} a^L (\frac{-1}{b})^{7-L}$ and we know that $x^L-(\frac{1}{x^2})^L=x^{-5}$.\\
    It's trivial that $L=3$. Thus, the coefficient is $\binom{7}{3} a^3 (\frac{-1}{b})^{7-3}$.\\
    Equating them: $\binom{7}{3} a^3 (\frac{-1}{b})^{7-3} = \binom{4}{3} a^3 (\frac{70}{27b})\\
    \iff \binom{7}{3} (\frac{1}{b})^{4} = \binom{4}{3} (\frac{70}{27b})\\
    \iff \frac{7!}{4! \cdot 3!} = \frac{4!}{3!} (\frac{70b^3}{27})\\
    \iff b^3=\frac{7*6*5*\cancel{4!}}{\cancel{4!} \cdot \cancel{3!}} \cdot \frac{\cancel{3!}}{4!} \cdot \frac{27}{70}\\
    \iff b^3=\frac{27}{8}\\
    \iff b=\frac{3}{2}\\
    \iff 2b=3$
\end{proof}
Here is an example for you to try.
\begin{example}
    (JEE adv 2013) The coefficient of three consecutive terms of $(1+x)^{n+5}$ are in the ratio $5:10:14$ then what is the value of $n?$
\end{example}
\section{Common Expansions and Factorizations}
Below are some expansions/factorizations which occur quite often. While you can easily prove them using binomial theorem or by expansion. I think you may already know most of them!
$(x+y)^2=x^2+2xy+y^2$\\
$(x-y)^2=x^2-2xy+y^2$\\
$(x+y)^2=(x-y)^2 + 4xy\\$
$x^2-y^2=(x-y)(x+y)\\$
$(x+y+z)^2=x^2+y^2+z^2+2(xy+yz+zx)\\$
$(x+y)^3=x^3+3x^2y+3xy^2+y^3\\$
$(x-y)^3=x^3-3x^2y+3xy^2-y^3\\$
$x^3-y^3=(x-y)(x^2+xy+y^2)\\$
$x^3+y^3=(x+y)(x^2-xy+y^2)$
This knowledge makes us quite powerful. Let's use that power now.\\
\begin{example}
Simplify $a^2+b^2+c^2-ab-bc-ca$
\end{example}
\begin{proof}
    [Solution]
    $a^2+b^2+c^2-ab-bc-ca$\\
    $= \frac{1}{2}(2a^2+2b^2+2c^2-2ab-2bc-2ca)$\\
    $= \frac{1}{2}(a^2-2ab+b^2+b^2-2bc+c^2+c^2-2ca+a^2)$\\
    $= \frac{1}{2}((a-b)^2+(b-c)^2+(c-a)^2)$
\end{proof}
Now that we have unlocked the power of algebraic expansion, we can use them to clean some dirty calculations:\\
\begin{example}
    Calculate $\frac{(2020^2-20100)(20100^2-100^2)(2000^2+200100)}{2010^6-10^6}$
\end{example}
\begin{proof}
[Solution]
    KEEP YOUR CALCULATOR DOWN. With our new found powers, the  question will shrivel before your eyes.\\
    $\frac{(2020^2-20100)(20100^2-100^2)(2000^2+20100)}{2010^6-10^6}\\
    = \frac{10^2(202^2-201)10^4(201^2-1)10^2(200^2+201}{10^6(201^6-1)}\\
    = \frac{10^8(202^2-201)(201^2-1)(200^2+201}{10^6(201^6-1}\\$
    Notice that the question has most of the terms somewhat close to $201$. Let $a=201$
    $\therefore \frac{10^8(202^2-201)(201^2-1)(200^2+201}{10^6(201^6-1}\\
    = \frac{10^2((a+1)^2-a)(a^2-1)((a-1)^2+a)}{a^6-1}\\
    = \frac{10^2(a^2+1+a)(a^2-1)(a^2+1-a)}{a^6-1}\\
    = \frac{10^2(a^2+1)^2-a^2)(a^2-1)}{a^6-1}\\
    = \frac{10^2(a^4+a^2+1)(a^2-1)}{a^6-1}\\
    = \frac{10^2\cancel{(a^6-1)}}{\cancel{a^6-1}}\\
    =10^2=100$
\end{proof}
Now you try:\\
\begin{example}
    Calculate $\sqrt{(500)(501)(502)(503)+1}$
\end{example}
\section{More Factorization Tricks}
While we can solve linear equations with ease, and we'll learn how to solve a quadratic(You may already know that); solving higher power equations is not that nice. Hence, we try to break them into smaller, more nicer equations. Some methods of the same are shown here. Up first: A generalization of an identity.
\begin{theorem}
    $x^n-y^n=(x-y)(x^{n-1}+x^{n-2}y+\dots +xy^{n-2}+y^{n-1}$\\
    NOTE: This happens for all natural values of $n$ and the sign in the second bracket is all positive.
\end{theorem}
\begin{theorem}
    $x^{2n+1}+y^{2n+1}=(x+y)(x^{2n}-x^{2n-1}y+\dots-xy^{2n-1}+y^{2n})$\\
    NOTE: This only happens for odd powers, and the sign in second bracket alternates.
\end{theorem}
We can prove the first simply by expanding. The second simply follows by $y \rightarrow -y$.\\
\begin{proof}
    $(x-y)(x^{n-1}+x^{n-2}y+\dots +xy^{n-2}+y^{n-1})\\
    = x^n+\cancel{x^{n-1}y} + \dots +\cancel{x^2y^{n-2}}+\cancel{xy^{n-1}} - \cancel{x^{n-1}y}-\cancel{x^{n-2}y^2} - \dots -\cancel{xy^{n-1}}-y^n\\
    = x^n-y^n$
\end{proof}
We will now talk about my favorite factorization trick, or more accurately, Simon’s Favorite Factoring Trick(SFFT).\\
The Simon here refers to Simon Rubinstein-Salzedo, the director of Euler's Circle, a very prestigious math program. The legend goes that he was one of the earliest people posting on AOPS. Richard Rusczyk hired him as a teaching assistant after he graduated high school. During this time, he solved a question using the method we are gonna talk about. He couldn't remember its name, and was going to say "Using my favorite factoring trick." but instead said, "Using Simon's favorite factoring trick."\\
Richard liked the name so much that he refereed to it as Simon’s Favorite Factoring Trick in the critically acclaimed Art of Problem Solving books. And the name stuck.\\
Here is what he says about this:\\
\begin{quote}
    It is very strange to me that my greatest claim to fame in life is a single forum post I made when I was 18 years old. I think I’ve done much better things in my life, especially running Euler Circle and trying to revolutionize gifted mathematics education, giving strong students an opportunity to get a dignified and challenging education that no one else is willing to offer them. However, it does seem that people find my factoring trick memorable with my name on it, so it seems to have done a small part in improving students’ problem-solving abilities; I’m glad about that. The really weird part, though, is that my factoring trick has made me something of a celebrity among math contest kids: they sometimes ask me for my autograph when I run math circles or show up at competitions. I certainly don’t deserve that
\end{quote}
So that's the story. Now let's come to the actual trick.
\begin{theorem}
    [SFFT]
    $xy+kx+ly=C \Rightarrow (x+l)(y+k)=C+kl$
\end{theorem}
\begin{proof}
[Line of Thought]
    $xy+kx+ly=C\\
\iff x(y+k)+ly=C\\
\iff x(y+k)+ly+kl=C+kl\\
\iff x(y+k)+l(y+k)=C+kl\\
\iff (x+l)(y+k)=C+kl$
\end{proof}
This may seem harmless enough, but it can take down all sorts of questions. Case in point:\\
\begin{example}
(AIME 1987) $m, n$ are integers such that $m^2 + 3m^2n^2 = 30n^2 + 517$. Find $3m^2n^2$.
\end{example}
\begin{proof}
    [Solution]
    Moving things around and Dividing by three will give us the the Simon form in $m^2,n^2$\\
    $ m^2n^2+\frac{m^2}{3}-10n^2=\frac{517}{3}\\
    \iff (m^2-10)(n^2+1/3)=\frac{507}{3}\\
    \iff (m^2-10)(3n^2+1)=507$\\
    This clearly limits our values as $(m,n)$ are integers and $507=1*507=3*169=13*39$. This forces us to $3n^2+1=13 \iff n=2$ and therefore $m=7$. The question wants us to return $3m^2n^2=3*2^2*7^2=3*4*49=588$
\end{proof}
A commonly occurring but hard to notice factorization is the Sophie Germain identity.
\begin{theorem}
    [Sophie Germain identity]
    $x^4+4y^4=(x^2+2y^2-2xy)(x^2+2y^2+2xy)$
\end{theorem}
This is a result of a process called completing the square. \\
\begin{proof}
    [Line of Thought]
    We know that $a^2+2ab+b^2=(a+b)^2$. Keeping that in mind,\\
    $x^4+4y^4=(x^2)^2+(2y^2)^2$\\
    $= (x^2)^2+(2y^2)^2 + 2(x^2)(2y^2)- 2(x^2)(2y^2)$\\
    We are doing this to use the identity of $(a+b)^2$\\
    $=(x^2+2y^2)^2-(2xy)^2$\\
    $=(x^2+2y^2-2xy)(x^2+2y^2+2xy)$
\end{proof}
The method of thinking here is more useful rather than the identity. We will also use it again in this chapter to unlock an even more powerful theorem. Here is a token use of Sophi\\
\begin{example}
    Prove that $x^4+1$ is composite for all integer values of $x$ such that $|x|>1$.
\end{example}
\begin{proof}
    What would other wise be a headache with number theory techniques becomes a mild embarrassment using Sophie Germain.\\
    $x^4+1\\
    = x^4 + 4*1^4\\
    =(x^2+2*1^2-2x)(x^2+2*1^2+2x)\\
    =(x^2+2-2x)(x^2+2+2x)\\$
    The only way for $x^4+1$ to be prime is if one of these is equal to 1. That occurs if and only if either $x^2-2x+2=1 \iff x^2-2x+1=0 \iff (x-1)^2=0 \iff x=1$ or $x^2+2x+2=1 \iff x^2+2x+1=0 \iff (x+1)^2=0 \iff x=-1$ which are both not possible as $|x|>1$. Hence, proved.\\ 
\end{proof}
The final factorization(which we have already seen in an example) we should keep in mind is(also a result of completing the square):
\begin{theorem}
    $x^4+x^2+1=(x^2+x+1)(x^2-x+1)$
\end{theorem}
While there are a lot more factorization which can be used, these are the most common(and almost everything else is derived from these or can be found only on expansion). You'll see there power as the chapter progresses.
\section{Quadratic Equations}
An equation of the form $ax^2+bx+c=0$ is known as a quadratic. It has many real life uses and comes up more than one expects. We solved one just above, but not all of them are that friendly. We have three common methods to solve quadratics.
\subsection{Splitting the middle term}
If $a=1$ and you can find two number, $\alpha$ and $\beta$ such that $\alpha + \beta = b$ and $\alpha \cdot \beta = c$, then:\\
$x^2+bx+c=0 \Rightarrow x^2+\alpha x +\beta x + \alpha \beta = 0\\
\Rightarrow x(x+\alpha)+\beta (x+\alpha)=0\\
\Rightarrow (x+\beta)(x+\alpha)=0\\
\Rightarrow x= -\beta$ or $-\alpha$\\
If $a \neq 1$ then either we can divide the equation by $a$ and do the above or look for $\alpha$ and $\beta$ such that $\alpha + \beta = b$ and $\alpha \cdot \beta = ac$ and then factorize. 
\begin{example}
    $20x^2+51x+22=0\\
    20x^2+40x+11x+22=0\\
    20x(x+2)+11(x+2)=0\\
    (x+2)(20x+11)=0$
\end{example}
\subsection{Completing the Square}
\begin{example}
    $2x^2+7x-4=0$ Here we can see that factorization is quite difficult. So we'll complete the square.\\
    $x^2+\frac{7}{2}x-2=0\\$ We know that $(x+a)^2=x^2+2ax+a^2$ lets create that over here.\\
    $x^2+2\frac{7}{4}x+\frac{49}{16}-\frac{49}{16}-2=0\\
    (x+\frac{7}{4})^2=\frac{49}{16}+2\\
    (x+\frac{7}{4})^2=\frac{49+32}{16}=\frac{81}{16}\\
    x+\frac{7}{4}=\frac{\mp 9}{4}\\
    x=\frac{2}{4}$ or $\frac{-16}{4}\\
    x=\frac{1}{2}$ or $-4$
\end{example}
As you might have noticed, this is literally SFFT with $x=y$. But for the times when we don't wish to use even the smallest part of our brian:
\subsection{Quadratic formula}
Using completing the square on the general equation,
$ax^2+bx+c=0\\
x^2+\frac{b}{a}x+\frac{c}{a}=0\\
x^2+2\frac{b}{2a}x+\frac{b^2}{4a^2}-\frac{b^2}{4a^2}+\frac{c}{a}=0\\
(x+\frac{b}{2a})^2=\frac{b^2}{4a^2}-\frac{c}{a}\\
(x+\frac{b}{2a})^2=\frac{b^2-4ac}{4a^2}\\
x+\frac{b}{2a}=\frac{\pm \sqrt{b^2-4ac}}{2a}\\
x=\frac{-b \pm \sqrt{b^2-4ac}}{2a}$\\
This is the quadratic formula for every single equation. While a bit messy, it solves the complicated equations the other methods can't.
\begin{theorem}
   For $ax^2+bx+c=0$
   $x=\frac{-b \pm \sqrt{b^2-4ac}}{2a}$
\end{theorem}
The part inside the square root, $b^2-4ac$ is known as the discriminant and decides the nature of roots.\\
If the discriminant is positive, we have 2 real roots; if it is zero, one real root and if it is negative no real roots(the roots are then imaginary. We'll talk about them later).\\
If the discriminant a square number and $b$ and $2a$ are rational, then the roots are rational. If it is not a square number, and $b$ and $2a$ are rational, the roots are irrational.\\
\section{Vieta's Formula}
The last section was us finding the roots using the coefficients, now we'll use the roots to find the coefficients.\\
Let $\alpha$ and $\beta$ be the roots of $ax^bx+c=0$, then:\\
$\alpha + \beta \\
= \frac{-b + \sqrt{b^2-4ac}}{2a}+\frac{-b - \sqrt{b^2-4ac}}{2a}\\
=\frac{-2b}{2a}\\
=\frac{b}{a}$\\
and also, $\alpha \cdot \beta\\
= \frac{-b + \sqrt{b^2-4ac}}{2a} \cdot \frac{-b - \sqrt{b^2-4ac}}{2a}\\
=\frac{b^2-b^2+4ac}{4a^2}\\
=\frac{c}{a}$\\
These are known as Vieta's formula. Another derivation of them happens simply using the splitting of middle term. It's quite trivial. Just divide the equation by $a$ and the rest is by definition of roots.
\begin{theorem}
    For an equation, $ax^2+bx+c=0$ \\
    Sum of roots is $\frac{-b}{a}$ and product of roots is $\frac{c}{a}$ 
\end{theorem}
However, Vieta can be generalized for higher degree polynomials as well.
\begin{theorem}
    For $ax^n+bx^{n-1}+cx^{n-2}\dots=0$ and the roots being $r_1, r_2, r_3 \dots r_n$,\\
    $\sum^n_{k=1}r_k=\frac{-b}{a}\\
    \sum^n_{k,l=1}=\frac{c}{a}\\
    \sum^n_{k,l,m}=\frac{-d}{a}\\
    \vdots$\\
    Note: We are basically taking sum of terms first one at a time, then two at a time(all possible pair of two terms) and then three at a time(all possible triplets of three terms) and so on, till we reach the product.
\end{theorem}
The proof will require the help of one additional theorem...
\section{The Fundamental Theorem of Algebra}
\begin{theorem}
    For any polynomial $ax^n+bx^{n-1}+cx^{n-2}\dots=0$ and the roots being $r_1, r_2, r_3 \dots r_n$, we can say\\
    $ax^n+bx^{n-1}+cx^{n-2}\dots=0$\\
    $\Rightarrow a(x-r_1)(x-r_2)(x-r_3)\dots(x-r_n)=0$
\end{theorem}
The above follows by the definition of being a root. We can now prove the generalized Vieta.\\
\begin{proof}
    Let  $P(x) = a_n (x-r_1)(x-r_2) \cdots (x-r_n)$\\
    Now for some $x^k$ we'll need to choose some $k$ of the $n$ factors to give us the $x$ and $n-k$ to give us the coefficient. The thing is that we can do this in all the ways $n-k$ can be chosen from $n$.\\
    So the total coefficient will end being $-1^{n-k}a_n$ times the sum of the product of $n-k$ roots in every which way(let's call it $S_{n-k}$. So we can say $-1^{n-k}aS_{n-k}$ is the cofficient of $x^k$ which we'll denote as $c_k$.\\
    This leads to $S_{n-k}=-1^{n-k} \frac{c_k}{a} \iff S_k=-1^k \frac{c_{n-k}}{a}$. Which is the generalized Vieta.\\
\end{proof}
Another theorem which is a consequence of the same is:
\begin{theorem}
    When $P(x)$ is divided by $x-r$ the remainder is $P(r)$
\end{theorem}
\begin{proof}
    Let $x-a$ divide $P(x)$ and there exists a quotient $Q(x)$ and remainder $R(x)$ such that\[P(x) = (x-a) Q(x) + R(x)\].\\
    We need to notice that $\deg R(x) < \deg (x-a)=1$ as if had equal or higher degree, we could divide it further. \\
    Now as $\deg (x-a)$ is 1, $\deg R(x)$ is 0, or it is a constent. Let this constant be $r$. We may substitute this into our original equation and rearrange to yield\[r = P(x) - (x-a) Q(x).\]\\
    When $x = a$, this equation becomes $r = P(a)$. Hence, the remainder upon diving $P(x)$ by $x-a$ is equal to $P(a)$.
\end{proof}
This short theorem might seem rather unremarkable but has a variety of uses. Case in point:\\
\begin{example}
    Given the cubic $f(x)=x^3+x+1$, $g(x)$ is another cubic whoose roots are the square of the roots of $f(x)$. Given $g(0)=-1$, find $g(9)$
\end{example}
\begin{proof}
    [Solution]
    Let $f(x)=(x-a)(x-b)(x-c)$ making $a,b,c$ roots $f(x)$.\\
    Therefore, $g(x)=k(x-a^2)(x-b^2)(x-c^2)$\\
    Taking $x=0$,\\
    $g(x)=-ka^2b^2c^2=-1 \iff ka^2b^2c^2=1$\\
    We know $abc=1$ using vieta on $f(x)$.\\
    $\therefore k*1=1 \iff k=1\\
    \therefore g(9)=(9-a^2)(9-b^2)(9-c^2)\\
    =(3+a)(3-a)(3+b)(3-b)(3+c)(3-c)$\\
    We can consider $f(3)=(3-a)(3-b)(3-c)$ and $-1f(-3)=-1(-3-a)(-3-b)(-3-c)=-1^3(-3-a)(-3-b)(-3-c)=(3+a)(3+b)(3+c)$\\
    $\therefore g(9) = -f(3)*f(-3)=899$
\end{proof}
However, there is even a shorter way out...\\
\begin{proof}
    [Alternate solution]
    As the squares of roots of $f(x)$ are the roots of $g(x)$, it is of the form $g(x)=kf(\sqrt{x})$.\\
    So we have: $x^{3/2}+x^{1/2}+1=0\\
    \iff x^{1/2}(x+1)=-1\\
    \iff x(x^2+2x+1)=1\\
    \iff x^3+2x^2+x-1=0$\\
    Hence, $g(x)=k(x^3+2x^2+x-1)$. As $g(0)=-1$, $k=1$.\\
    Therefore, $g(x)=x^3+2x^2+x-1$. We can now simply compute $g(9)=899$
\end{proof}
Another theorem, which provides a valid but tedious and messy way to find rational roots i:\\
\begin{theorem}
    Given an integer polynomial $P(x)$ with leading coefficient $a_n$ and constant term $a_0$, if $P(x)$ has a rational root $r = p/q$ in lowest terms, then $p$ divides $a_0$ and $q$ divides $a_n$
\end{theorem}
While not ideal, this theorem is sometimes the only weapon of attack we have against more complicated polynomials. Once we find a single root, we can use the factor theorem and divide the polynomial into simpler parts. Below is a proof for the same:\\
\begin{proof}
    Let $\frac{p}{q}$ be in its simplest form and a rational root of $P(x) = a_n x^n + a_{n-1} x^{n-1} + \cdots + a_0$, where every $a_r$ is an integer; \\
    Since $\frac{p}{q}$ is a root of $P(x)$,
    \[0 = a_n \left(\frac{p}{q}\right)^n + a_{n-1} \left(\frac{p}{q}\right)^{n-1} + \cdots + a_1 \left(\frac{p}{q}\right) + a_0.\]\\
    Multiplying by $q^n$ yields
    \[0 = a_n p^n + a_{n-1} p^{n-1} q + \cdots + a_1 p * q^{n-1} + a_0 q^n.\]\\
    We need to notice every term other than $a_0 q^n$ is divisible by $p$ and so is zero. Which means $a_0 q^n$ is also divisible by $p$. However, as $p$ and $q$ have no common factors(as $\frac{p}{q}$ is in its simplest form), $\therefore a_0$ is divisible by $p$. Simlerly, $a_n$ is divisible by $q$.
\end{proof}
We'll give it a whirl with this token example:\\
\begin{example}
    Find all rational roots of $6x^3 + x^2 - 19x + 6$
\end{example}
\begin{proof}
    [Solution]
    The possible values of $p$ are $-6,-3,-2,-1,1,2,3,6$ and possible values of $q$ are $1,2,3,6$(if both are negative, it cancels and becomes both positive)\\
    Hence, $\frac{p}{q}$ has possible values, $-6,-3,-2,-1,\frac{-3}{2}, \frac{-1}{2}, \frac{-2}{3}, \frac{-1}{3}, \frac{-1}{6}, \frac{1}{6}, \frac{1}{3}, \frac{2}{3}, \frac{1}{2}, \frac{3}{2}, 1, 2, 3, 6$\\
    Of all the eighteen values, we'll check the simplest ones first. $-2$ is a root. Therefore, our polynomial is divisible by $x+2$ which will leave a a quadratic which we can factorize quite easily.\\
    $\therefore 6x^3+x^2-19x+6=(6x^2-11x+3)(x+2)\\
    =(2x-3)(3x-1)(x+2)$\\
    Hence the rational roots are: $\frac{3}{2}, \frac{1}{3}$ and $2$.
\end{proof}
\section{Newton's Sums}
Next, we'll discuss Newton Sums. We can add roots being multiplied to other roots using Vieta. What about squares of all the roots? What about the cubes? This solves questions which are seemingly complex using Vieta's very easily.\\
\begin{theorem}
    For $P(x)=a_nx^n+a_{n-a}x^{n-a}+\dots+a_1x+a_0$ with roots $r_1, r_2, r_3 \dots, r_n$,\\
    Let: $S_1=r_1+r_2+\dots r_n\\
    S_2=r_1^2+r_2^2+\dots r_n^2\\
    \vdots\\
    S_k=r_1^k+r_2^k+\dots r_n^k$\\
    then the following will hold true:\\
    $a_nS_1+a_{n-1}=0\\
    a_nS_2+a_{n-1}S_1+2a_{n-2}=0\\
    \vdots$
\end{theorem}
Basically what the theorem says is:
\begin{enumerate}
    \item Start with a $S_k$ value and multiply by it by the leftmost polynomial coefficient.
    \item Then, multiply $S_{k-1}$ by the polynomial’s coefficient right after it.
    \item Continue doing so and summing the products until $a_{k-i}$ becomes 0 in which case we simply add the last term and stop
    \item Set your final sum of terms to be equal to 0
\end{enumerate}
\begin{proof}
\small % Reduce font size

Let $\alpha,\beta,\gamma,...,\omega$ be the roots of a given polynomial $P(x)=a_nx^n+a_{n-1}x^{n-1}+..+a_1x+a_0$. Then, we have that
\[P(\alpha)=P(\beta)=P(\gamma)=...=P(\omega)=0\]

Thus,
\[
\begin{cases}
a_n\alpha^n+a_{n-1}\alpha^{n-1}+...+a_0=0\\
a_n\beta^n+a_{n-1}\beta^{n-1}+...+a_0=0\\
\vdots\\
a_n\omega^n+a_{n-1}\omega^{n-1}+...+a_0=0
\end{cases}
\]

Multiplying each equation by $\alpha^{k-n},\beta^{k-n},...,\omega^{k-n}$, respectively,
\[
\begin{cases}
a_n\alpha^{n+k-n}+a_{n-1}\alpha^{n-1+k-n}+...+a_0\alpha^{k-n}=0\\
a_n\beta^{n+k-n}+a_{n-1}\beta^{n-1+k-n}+...+a_0\beta^{k-n}=0\\
\vdots\\
a_n\omega^{n+k-n}+a_{n-1}\omega^{n-1+k-n}+...+a_0\omega^{k-n}=0
\end{cases}
\]

\[
\begin{cases}
a_n\alpha^{k}+a_{n-1}\alpha^{k-1}+...+a_0\alpha^{k-n}=0\\
a_n\beta^{k}+a_{n-1}\beta^{k-1}+...+a_0\beta^{k-n}=0\\
\vdots\\
a_n\omega^{k}+a_{n-1}\omega^{k-1}+...+a_0\omega^{k-n}=0
\end{cases}
\]

now taking the sum,
\[
a_n\underbrace{(\alpha^k+\beta^k+...+\omega^k)}_{P_k}+a_{n-1}\underbrace{(\alpha^{k-1}+\beta^{k-1}+...+\omega^{k-1})}_{P_{k-1}}+\]\\
\[+a_0\underbrace{(\alpha^{k-n}+\beta^{k-n}+...+\omega^{k-n})}_{P_{k-n}}=0
\]

Therefore,
\[a_nP_k+a_{n-1}P_{k-1}+a_{n-2}P_{k-2}+...+a_0P_{k-n}=0.\]

\end{proof}
This is a monsterous proof. No wonder Newton got it during the black plague.\\
Let's now use the theorem as a example:\\
\begin{example}
    (USAMO 1973) Determine all the roots, real or complex, of the system of simultaneous equations 
    $x+y+z=3$,
    $x^2+y^2+z^2=3$,
    $x^3+y^3+z^3=3$.
\end{example}
\begin{proof}
    [Solution]
    Let $x,y,z$ be the roots of some cubic polynomial $f(t)=at^3+bt^2+ct+d$.\\
    Let $x+y+z=S_1=3$, $x^2+y^2+z^=S_2=3$ and $x^3+y^3+z^3=S_3=3$\\
    Then using newton sums: $3a+b=3 ; 3a+3b+2c=0; 3a+3b+3c+3d=0$\\
    as $b=-3a$ and $a+b+c+d=0$,
    $\therefore 3a+3b+2c=0 \iff 2b+2c=0 \iff b+c=0 \iff b=-c$\\
    $\therefore a=-d$\\
    $\therefore f(t)= at^3-3at^2+3at-a$\\
    Now we need to find roots of this equation:\\
    $at^3-3at^2+3at-a=0\\
    \iff t^3-3t^2+3t-1=0
    \iff (t-1)^3=0\\
    \iff t=1$\\
    Thus, the only possible solution is $x=y=z=1$
\end{proof}
\section{Reciprocal Relations}
The given relations tend to come up time to time in speed solving competitions(where you should remember them) and admission tests. There only other use occurs in the simplification of even symmetric polynomials(see below). So it is recommended to memorize them by simply deriving them yourself. Do it the next day as well.\\
You will now not forget them for rest of your days.
\begin{theorem}
    If $x+\frac{1}{x}=a$ then:\\
    $x^2+\frac{1}{x^2}=a^2-2\\
    x^3+\frac{1}{x^3}=a^3-3a\\
    x^4+\frac{1}{x^4}=(a^2-2)^2-2$
\end{theorem}
\begin{theorem}
    If $x-\frac{1}{x}=a$ then:\\
    $x^2+\frac{1}{x^2}=a^2+2\\
    x^3-\frac{1}{x^3}=a^3+3a\\
    x^4+\frac{1}{x^4}=(a^2+2)^2+2$
\end{theorem}
\section{Some Special Polynomials}
We are going to talk about a few special polynomials now. These tend to appear a lot and we should be alert when they do as they are like Unicorn sightings, scary at first but wonderful on retrospection.
\subsection{Symmetric Polynomial}
\begin{definition}
    A polynomial $P(x)=a_nx^n+a_{n-1}x^{n-1}\dots+a_1x+a_0$ is symmetric if and only if:\\
$a_n=a_0\\
a_{n-1}=a_1\\
\vdots\\
a_{n-k}=a_k$\\
\end{definition}
If it is of an even degree, it can be solved with the following algorithm:\\
\begin{theorem}
    [Algorithm]
    \begin{enumerate}
    \item Divide by $x^{n/2}$
    \item Group $x^k$ with $\frac{1}{x^k}$
    \item Make the substitution $t=x+\frac{1}{x}$
    \item Solve the simplified polynomial.
\end{enumerate}
\end{theorem}
Let's solve an example for more clarity.
\begin{example}
    The real number $x$ satisfies the equation $x+ \frac{1}{x} = \sqrt{5}$. What is the value of $x^11-7x^7+x^3$?
\end{example}
\begin{proof}
    [Solution]
    Before you complain that the given equation is neither symmetric nor of even degree, take note that it is neither symmetric nor of even degrees.\\
    We can see that $a_0 = a_1 = a_2=0$ and hence I can keep $a_14=a_13=a_12=0$. Making it both even and symmetric.\\
    $\therefore x^11-7x^7+x^3\\
    =0x^14+x^11-7x^7+x^3+0$ Now let's divide by $x^7$\\
    $=x^4-7+\frac{1}{x^4}$ using the reciprocal relations,\\
    $=(\sqrt{5}^2-2)^2-2-7\\
    =0$
\end{proof}
\subsection{Reciprocal Roots}
\begin{theorem}
    If the roots of polynomial $P(x)=a_nx^n+a_{n-1}x^{n-1}\dots+a_1x+a_0$ are $r_1,r_2,\dots r_n$ then the roots of $a_0x^n+a_1x^{n-1}\dots+a_{n-1}x+a_n$ are $\frac{1}{r_1}, \frac{1}{r_2}, \dots \frac{1}{r_n}$\\
\end{theorem}
No proof is required as it is quite trivial to see using Vieta.
\subsection{Adding to all roots}
\begin{theorem}
    If the roots of polynomial $P(x)=a_nx^n+a_{n-1}x^{n-1}\dots+a_1x+a_0$ are $r_1,r_2,\dots r_n$ then the roots of $a_n(x-k)^n+a_{n-1}(x-k)^{n-1}\dots+a_{1}(x-k)+a_0$ are $r_1+k, r_2+k, \dots r_n+k$
\end{theorem}
This doesn't require a proof as it follows from the definition of root.\\
Now let's get started with some exercises.
\begin{xcb}{Exercises}
\begin{enumerate}
\item (AIME) Solve for positive value of x:\\
$\frac{1}{x^2-10x-29}+\frac{1}{x^2-10x-45}-\frac{2}{x^2-10x-69}=0$
\item (AIME) Let (a, b, c) be the real solution of the system of equations $x^3 - xyz = 2, y^3 - xyz = 6,
z^3 - xyz = 20$. The greatest possible value of $a^3 + b^3 + c^3$ can be written in the form $\frac{m}{n}$,
where m and n are relatively prime positive integers. Find $m + n$
\item (AIME) Find $x^2 + y^2$ if x and y are positive integers such that\\
$xy + x + y = 71$,\\
$x^2y + xy^2 = 880$.
\item (AIME) The equation $2^{333x-2} + 2^{111x+2} = 2^{222x+1} + 1$ has three real roots. Given that their
sum is $m/n$ where m and n are relatively prime positive integers, find $m + n$.
\item (AIME) What is the product of the real roots of the equation $x^2 + 18x + 30 = 2 \sqrt{x^2 + 18x + 45}$?
\item If the roots of the quadratic $3x^2 + 6x - 5$ are r and s, find $r^3 + s^3$.
\item If r, s, and t are roots of the cubic $x^3 - 6x - 5 = 0$, find $r^3 + s^3 + t^3$?
\item (AIME) Suppose that the roots of $x^3 + 3x^2 + 4x - 11 = 0$ are a, b, and c, and that the roots of $x^3 + rx^2 + s^x + t = 0$ are $a + b, b + c,$ and $c + a$. Find t.?
\item Solve for real roots of $10000x^4 - 5000x^3 + 825x^2 - 50x + 1 = 0$
\item Find the products of the solutions for:\\
$\sqrt{5|x|+8}=\sqrt{x^2-16}$
\item (AIME) The only real root of $8x^3-3x^2-3x-1=0$ can be written as $\frac{\sqrt[3]{a}+\sqrt[3]{b}+1}{c}$ in simplest form. What is $a+b+c$
\item (AMC 12) A rectangular floor measures $a$ by $b$ feet, where $a$ and $b$ are positive integers with $b > a$. An artist paints a rectangle on the floor with the sides of the rectangle parallel to the sides of the floor. The unpainted part of the floor forms a border of width $1$ foot around the painted rectangle and occupies half of the area of the entire floor. How many possibilities are there for the ordered pair $(a,b)$? 
\item(AMC 12) For certain real numbers $a$, $b$, and $c$, the polynomial\[g(x) = x^3 + ax^2 + x + 10\]has three distinct roots, and each root of $g(x)$ is also a root of the polynomial\[f(x) = x^4 + x^3 + bx^2 + 100x + c.\]What is $f(1)$?
\item(IOQM 2023) A positive integer $m$ has the property that $m^2$ is expressible in the form $4n^2-5n+16$ where n is an integer (of any sign). Find the maximum possible value of $|m - n|$.
\item (AIME) The polynomial $P(x)$ is cubic. What is the largest value of $k$ for which the polynomials $Q_1(x) = x^2 + (k-29)x - k$ and $Q_2(x) = 2x^2+ (2k-43)x + k$ are both factors of $P(x)$?
\item(CAMO 1998) For what real values of $k$ do $1988x^2+kx+8891$ and $8891x^2 + kx + 1988$ have a common zero?
\item (AIME 2018) A real number $a$ is chosen randomly and uniformly from the interval $[-20, 18]$. The probability that the roots of the polynomial $x^4 + 2ax^3 + (2a - 2)x^2 + (-4a + 3)x - 2$ are all real can be written in the form $\dfrac{m}{n}$, where $m$ and $n$ are relatively prime positive integers. Find $m + n$.
\item (AIME 2001) Find the sum of the roots, real and non-real, of the equation $x^{2001}+\left(\frac 12-x\right)^{2001}=0$, given that there are no multiple roots.
\item (AIME 1996) Suppose that the roots of $x^3+3x^2+4x-11=0$ are $a$, $b$, and $c$, and that the roots of $x^3+rx^2+sx+t=0$ are $a+b$, $b+c$, and $c+a$. Find $t$.
\item (AIME 2005) The equation $2^{333x-2} + 2^{111x+2} = 2^{222x+1} + 1$ has three real roots. Given that their sum is $m/n$ where $m$ and $n$ are relatively prime positive integers, find $m+n.$
\item (AIME 1983) Suppose that the sum of the squares of two complex numbers $x$ and $y$ is $7$ and the sum of the cubes is $10$. What is the largest real value that $x + y$ can have?
\item (AIME 2015) Let $x$ and $y$ be real numbers satisfying $x^4y^5+y^4x^5=810$ and $x^3y^6+y^3x^6=945$. Evaluate $2x^3+(xy)^3+2y^3$.
\item (PRMO 2019) Let $f(x) = x^2 + ax + b$. If for all nonzero real $x$: \\
\[f(x+\frac{1}{x})=f(x)+\frac{1}{x}\]\\
If the roots of $f(x)$ are integers, what is the value of $a^2+b^2$?
\item $x+y+z=1\\
x^2+y^2+z^2=2\\
x^3+y^3+z^3\\$
Find $x^4+y^4+z^4$?
\item (INMO 1991) How many ordered triples (x, y, z) of real numbers satisfy the system of equations:\\
$x^2+y^2+z^2=9\\
x^4+y^4+z^4=33\\
xyz=-4$
\item (AIME 2016) Let $P(x)$ be a nonzero polynomial such that $(x-1)P(x+1)=(x+2)P(x)$ for every real $x$, and $\left(P(2)\right)^2 = P(3)$. Then $P(\tfrac72)=\tfrac{m}{n}$, where $m$ and $n$ are relatively prime positive integers. Find $m + n$.
\item (AIME 1983)What is the product of the real roots of the equation $x^2 + 18x + 30 = 2 \sqrt{x^2 + 18x + 45}$?
\item (USAMO 1984) In the polynomial $x^4 - 18x^3 + kx^2 + 200x - 1984 = 0$, the product of $2$ of its roots is $- 32$. Find $k$.
\item (RMO 2107) Let $P(x) = x^2 +\frac{x}{2}+b$ and $Q(x) = x^2 +cx+d$ be two polynomials with real coefficients such that $P(x)Q(x)=Q(P(x))$ for all real $x$. Find all real roots of $P(Q(x)) = 0$.
\item (PRMO 2018) Let $P(x) = a_0 + a_1x + a_2x^2 + \dots + a_nx^n$ be a polynomial in which $a_i$ is non-negative integer for each $i \in 0, 1, 2, 3, \dots, n$. If $P(1) = 4$ and $P(5) = 136$, what is the value of P(3)?
\item (PRMO 2018) Integers $a,b,c$ satisfy $a+b-c=1$ and $a^2+b^2-c^2=-1$. What is the sum of possible values of $a^2+b^2+c^2$
\item (CMI 2019) Find all roots of \\
$\frac{8^{x}+27^{x}}{12^{x}+18^{x}}=\frac{7}{6}$
\item (AIME 1986) The polynomial $1-x+x^2-x^3+\cdots+x^{16}-x^{17}$ may be written in the form $a_0+a_1y+a_2y^2+\cdots +a_{16}y^{16}+a_{17}y^{17}$, where $y=x+1$ and the $a_i$'s are constants. Find the value of $a_2$.
\end{enumerate}
\end{xcb}