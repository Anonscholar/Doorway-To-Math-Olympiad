
\chapter{Functional Equations}
Functional equations are equations which are defined on functions rather than in variables. These require surprising amount of intuition and guessing to solve(and then also sometimes come out wrong). However, this should not deter us from learning about them.\\
We will start of small and finally explore what are called 'monsters'. However, you may need to read some parts more than once to understand what is going on. So I request you to not move on until you are clear about a piece of text.\\
\section{Some definitions}
Let $f : A \to B$ be a function.
The set $A$ is called the \textbf{domain}, and $B$ the \textbf{co-domain}.
A couple definitions which will be useful:
\begin{definition}
	A function $f : A \to B$ is \textbf{injective} if $f(x) = f(y) \iff x=y$.
	(Sometimes also called \textbf{one-to-one}.)
\end{definition}
\begin{definition}
	A function $ : A \to B$ is \textbf{surjective} if for all $b \in B$,
	there is some $x \in A$ such that $f(x) = b$.
	(Sometimes also called \textbf{onto}.)
\end{definition}
\begin{definition}
	A function is \textbf{bijective} if it is both injective and surjective.
\end{definition}
In this chapter, by solve over $K$, we will mean find all functions $f : K \to K$ such that the equation holds for all inputs in $K$.\\
For example: \\
There’s a function from living humans to $\mathbb{Z}_{\geq 0}$ by taking every human to their age in years (rounded to the nearest integer). This function is not injective, because for example there are many people with age $20$. This function is also not surjective: no one has age $10000$.
There’s also a function taking every Indian citizen to their Aadhar card number, which we view as a function from citizens to $\mathbb{Z}_{\geq 0}$. This is also not surjective (no one has card number equal to $00\dots 3$), but it is injective (no two people have the same number).\\
This example should help remembering the functions.\\
\section{Basic Functional Equations}
\begin{example}
[Motivation Example]
(USAMO 2016) Find all functions $f:\mathbb{R}\rightarrow \mathbb{R}$ such that for all real numbers $x$ and $y$,\[(f(x)+xy)\cdot f(x-3y)+(f(y)+xy)\cdot f(3x-y)=(f(x+y))^2.\]
\end{example}
One of the most tricky motivating example of this book. But this one question will explore most of the possible configrations of this chapter.\\
We can get some insight by first trying out random standard vales. Like, let $y=0$, will give us:\\
$f(x)*f(x)+f(0)*f(3x)=f(x)^2\\
f(0)*f(3x)=0$\\
This means either $f(3x)=0 \implies f(x)=0$ or $f(0)=0$. As we have got one solution, we'll now try to see where $f(0)=0$ takes us. Let $x=0$\\
$f(0)*f(-3y)+f(y)*f(-y)=f(y)^2$ as $f(0)=0$\\
$\iff f(y)f(-y)=f(y)^2$ which clearly shouts $f(y)=f(-y)$\\
At this stage my intuition blurts that $f(x)=x^2$ which works for the given equation.\\
However, we can show that that is the case by simply taking $x-3y=3x-y \iff x=-y$ to cancel out the unsymmetrical terms in the question.\\
$f(2x)(f(x)-x^2+f(-x)-x^2)=f(0)^2$\\
We use $f(0)=0$ and $f(x)=f(-x)$\\
$(f(x)-x^2)f(4x)=0 \iff f(x)=x^2$ or $f(4x)=0$\\
This solves the question. However, we need to also prove that no other function satisfies this equation.\\
We do this by contradiction,\\
Assume to the contrary that $f(x)$ is such that $f(a)=0$ for $a \neq 0$ and $f(b)=b^2$.\\
We'll let $x-3y=a$ and $3x-y=b$\\
$(f(x)+xy)a^2=f(x+y)^2$\\
As $b$ is arbitrarily large as $f(b)=f(2b)=f(4b)=\dots=0$ as we have seen above, we can  claim $x,y> 0$ and therefore $f(x)+xy \geq xy > 0$ and hence, $f(x+y)^2 > 0$ which means $f(x+y)=(x+y)^2$\\
$(x+y)^4=(f(x)+xy)(x-3y)^2\\
\iff (x+y)^4 \leq (x^2+xy)(x-3y)^2\\
\iff (x+y)^4 < (x+y)^2(x-3y)^2\\
\iff (x+y)^4 < (x+y)^4$\\
Which is a contradiction and hence no such function exists.\\
Therefore the only such functions are $f(x)=0$ or $f(x)=x^2$\\
The last step of proving is called the pointwise trap and not doing it is the reason why despite being correct a lot of people lose marks in functional equations.\\
While we used a few tricks here, we'll learn a few more now.
\section{Some methods of solving}
\subsection{Forced cancellations} We'll explore this using the given example:\\
\begin{example}
[Motivating Example]
    Find all $f: \mathbb{R} \to \mathbb{R}$ such that:
    \[f(x^2+y)=f(x^{27}+2y)+f(x^4)\]
    for $x,y \in \mathbb{R}$
\end{example}
\begin{proof}
    [Solution]
    A little thinking can convince us that $f(x)=0$, however how do we go about proving it?\\
    Let's set $y$ such that: $x^2+y=x^{27}+2y$\\
    $\therefore y=x^2-x^{27}$\\
    Which will mean, $f(x^4)=0$, hence $f(x)=0$ for all positive reals.\\
    Now let's extend it to reals. Let $y=0$,\\
    $f(x^2)=f(x^{27})+f(x^4)\\
    \iff 0=f(x^{27})+0\\
    \iff f(x^{27})=0$ making it 0 for negative reals as well.
\end{proof}
\subsection{The FFF trick!}
Normally, $f(f(x))$ does more bad than good. But sometimes, we can see this as an opportunity and put another $f$. Let's see it in action:
\begin{example}
    Find all strictly increasing functions $f : \mathbb{Z} \to \mathbb{Z}$ such that $f(f(x)) = x + 2$ for all integers x
\end{example}
\begin{proof}
[Solution]
Let's add a $f$ on both the sides.\\
$f(f(f(x)))=f(x+2)\\
\iff f(x)+2=f(x+2)\\
\iff f(x)=x+k$\\
We put this in the original equation to solve for $k$\\
$f(f(x))=x+2\\
\iff f(x+k)=x+2\\
\iff x+2k=x+2\\
\iff k=1$\\
Thus, $f(x)=x+1$
\end{proof}
\subsection{Symmetry}
If the equation, or some parts of it are symmetric in terms of $x$ and $y$, we can interchange them for insight.
\begin{example}
Find all functions $f : \mathbb{R} \to \mathbb{R}$ such that\\
$xf(x) + y^2 + f(xy) = f(x + y)^2 - f(x)f(y)$\\
for all real numbers $x$ and $y$
\end{example}
\begin{proof}
    Let's swap for $x$ and $y$,\\
    $yf(y) + x^2 + f(xy) = f(x + y)^2 - f(x)f(y)$\\
What we need to notice is that the RHS is exactly the same. So subtracting the original equation from the new one:\\
$xf(x)-x^2=yf(y)-y^2$\\
It is rather obvious that the solution is $f(x)=x$
\end{proof}
Here I would also like to include a exam focused note: that functional equations over real numbers tend to have simpler solutions like $c, x, kx, kx+c, x^n, kx^n, kx^n+c$ and one can literally try all of them out to solve a complex one and just prove that it is the solution. Moreover, $n > 3$ is so rare that it is an abnormality in the space time continuum. Also in the worst case scenario, we may have a polynomial, so that one thing.\\
Is it correct or the ethical way to solve the problem? NO. Will Cauchy come in your dream and haunt you? Probably. But will it score marks? Defiantly.\\
While, this trick is not applicable to functional equations over integers or whole numbers as you have divisibility, and other things, most of which work because integers are ”discrete”. However,  strange functions, like ones that depend on mods or something are rare enough that you can just use the base forms here as well and still get the answer most of the time.\\
\begin{example}
(USAMO 2019)
    Let $\mathbb{N}$ be the set of positive integers. A function $f:\mathbb{N}\to\mathbb{N}$ satisfies the equation\[\underbrace{f(f(\ldots f}_{f(n)\text{ times}}(n)\ldots))=\frac{n^2}{f(f(n))}\]for all positive integers $n$. Given this information, determine all possible values of $f(1000)$.
\end{example}
\begin{proof}
    Assume $f(a) = f(b)$. Then 
\[f^{f(a)}(a) = f^{f(b)}(b)\]
since $f$ is applied the same number of times. Thus we get
\[a^2 = f^{f(a)}(a) = f^{f(b)}(b) = b^2\]
so $a = b$. Thus, $f(a) = f(b) \iff a = b$, so $f$ is injective.\\

Let $n = 1$. Then we get $f^{f(1)}(1) = 1$ and $f^{f(1)} = 1$. Now let $f(1) = k$.
We get
\[k^2 = f^{f(k)}(k) = f(f(k)) = f(k)f(1) = k\]
so $k = 1$, therefore $f(1) = 1$.\\
Now we use \textbf{Induction}\\
Claim: $f(n) = n$ for all odd numbers $n$.
(B) We have that $f(1) = 1$, so our base case is done. \\
(S) Now assume it is true for $n = 1, 3, \ldots, 2k - 1$, and let $n = 2k + 1$. If $f(f(n)) < n$, then $f(f(n)) = m$ for some odd number $m < n$. But then by injectivity, $f(n) = m$, so $n = m$, which is a contradiction. Similarly, if $f^{f(n)}(n) = m < n$, then $n = m$, which is another contradiction. Thus, we must have $f(f(n)) = f^{f(n)}(n) = n$. So our induction is complete.\\
From injectivity, we know $f(1000)$ can't be odd. Notice that in the above proof, $n$ being odd was important, and if it was even, we couldn't conclude anything. This makes it seem like $f(1000)$ could be any even number, and we only need to find one such function. Now consider the function
\[
f(n) =
\begin{cases}
2k & \text{if } n = 1000 \\
1000 & \text{if } n = 2k \\
n & \text{if } n \neq 2k, 1000 \\
\end{cases}
\]
This function was chosen since we want $f(f(n)) = n$, which would make the function $f$ satisfy the equation. Thus, the even numbers are all possible values of $f(1000)$.
\end{proof}
This is a complicated question, so I recommend reading the proof once more. It had taken me way longer than it should have to understand the proof.\\
\section{Cauchy's Functional Equations}
Cauchy's functional equations refers to certain set of functional equations, all which are plays on $f(x)+f(y)=f(x+y)$. Using them can simplify a lot of questions. I recommend memorizing them so as to not needing to derive them whenever they come up/\\
\begin{theorem}
[Cauchy's first functional equation]
    For  $f : \mathbb{Q} \to \mathbb{Q}$ such that:\\
    \[f(x)+f(y)=f(x+y)\]\\
where $f(x)$ is continuous if and only if $f(x)=kx$
\end{theorem}
The proof of the same is trivial via induction. I expect that you'll be able to do this.\\
\begin{theorem}
[Cauchy's second functional equation]
    For  $f : \mathbb{Q} \to \mathbb{Q}$ such that:\\
    \[f(x \cdot y)=f(x)+f(y)\]\\
where $f(x)$ is continuous if and only if $f(x)=k \ln{x}$
\end{theorem}
\begin{proof}
    We know that every $x=a^u$ for a unique $u\in \mathbb{R}$  for every $x,a$. This allows us to make a substitution:\\
    \[ f(a^u \cdot a^v)=f(a^u)+f(a^v)\]
    This transforms into the first Cauchy functional equation by defining $f(a^x)=g(x)$. This solves to give us $f(x)=k\ln{x}$
\end{proof}
\begin{theorem}
[Cauchy's third functional equation]
    For  $f : \mathbb{Q} \to \mathbb{Q}$ such that:\\
    \[f(x + y)=f(x)\cdot f(y)\]\\
where $f(x)$ is continuous and non-zero if and only if $f(x)=a^x$ where $a=f(1)$
\end{theorem}
\begin{proof}
    We can take $\log$ on both sides.\\
    $\log{f(x+y)}=\log{f(x)}+\log{f(y)}$\\
    Taking $\log{f(x)}=g(x)$, this transforms into the first Cauchy. We solve to get $f(x)=a^x$ where $a=f(1)$
\end{proof}
\begin{theorem}
[Cauchy's fourth functional equation]
    For  $f : \mathbb{Q} \to \mathbb{Q}$ such that:\\
    \[f(x \cdot y)=f(x)\cdot f(y)\]\\
where $f(x)$ is continuous and non zero if and only if $f(x)=x^k$
\end{theorem}
\begin{proof}
    This time taking $\log$ transforms to the second Cauchy. Which leads to $f(x)=x^k$
\end{proof}
\section{Checklist for Functional Equation Solving}
At the beginning of a problem:
\begin{itemize}
    \item Figure out what the answer is using the common solutions list.
    \item Plug in $x = y = 0, x = 0$ into the equation
    \item Plug in things that make lots of terms cancel.
    \item Look to see if the FFF trick can be used.
    \item Look to check if it is injective or surjective and if that can tell us something
    \item Look for symmetry(or breaks in that)
    \item Look for opportunities to use induction
    \item Look for ways to simplify it to Cauchy
    \item Make sure you don't fall in the pointwise trap
    \item Check if the solution actually works
\end{itemize}
Let's start the exercises now.
\begin{xcb}{Exercises}
\begin{enumerate}
\item (USAJMO 2015) Find all functions $f:\mathbb{Q}\rightarrow\mathbb{Q}$ such that\[f(x)+f(t)=f(y)+f(z)\]for all rational numbers $x<y<z<t$ that form an arithmetic progression. ($\mathbb{Q}$ is the set of all rational numbers.)
\item (USAMO 2002) Let $\mathbb{R}$ be the set of real numbers. Determine all functions $f : \mathbb{R} \rightarrow \mathbb{R}$ such that \[f(x^2 - y^2) = xf(x) - yf(y)\] for all pairs of real numbers $x$ and $y$.
\item Find all functions $f: \mathbb{R} \to \mathbb{R}$ satisfying: \[ f\left(\frac{x^2}{2} + y\right) = f\left(f(x) - y\right) + 4f(x)y \]
\item (IMO 2010) Find all function $f:\mathbb{R}\rightarrow\mathbb{R}$ such that for all $x,y\in\mathbb{R}$ the following equality holds
\[f(\left\lfloor x\right\rfloor y)=f(x)\left\lfloor f(y)\right\rfloor\]
\item (AHSME 1998) Let $f(x)$ be a function with the two properties:\\
(a) for any two real numbers $x$ and $y$, $f(x+y) = x + f(y)$, and\\
(b) $f(0) = 2.$\\
What is the value of $f(1998)?$
\item (AIME 1994) The function $f_{}^{}$ has the property that, for each real number $x,\,$
$f(x)+f(x-1) = x^2.\,$
If $f(19)=94,\,$ what is the remainder when $f(94)\,$ is divided by $1000$?
\item (BMO 1997) A non-negative integer $f(n)$ is assigned to each positive integer n in such a way that the following conditions are satisfied:\\
(a) $f(mn) = f (m) + f (n)$, for all positive integers $m$, and $n$;\\
(b) $f (n) = 0$ whenever the units digit of $n$ (in base $10$) is a $3$; and\\
(c) $f (10) = 0$.\\
Prove that $f (n) = 0$, for all positive integers $N$
\item (Putnam 1999, A1) Find polynomials $f(x)$, $g(x)$, and $h(x)$, if they exist, such that, for all $x$:
\[
|f(x)|-|g(x)|+h(x) =
\begin{cases}
    -1 & \text{if } x < -1 \\
    3x + 2 & \text{if } -1 \leq x \leq 0 \\
    -2x + 2 & \text{if } x > 0
\end{cases}
\]
\item (Russia 1988) The functions $f (x)$ and $g(x)$ are defined on the real axis so that they satisfy the following condition: for any real numbers $x$ and $y$, $f (x + g(y)) = 2x + y + 5$. Find an explicit expression for the function $g(x + f (y))$.
\item (IMO 1977) Let $f(n)$ be a function $f: \mathbb{N}^{+}\to\mathbb{N}^{+}$. Prove that if\[f(n+1) > f(f(n))\]for each positive integer $n$, then $f(n)=n$.
\item  (Russia 1991) Does there exist a function $F : \mathbb{N} \to \mathbb{N}$ such that for any natural number $x$,
$F (F (F ( \dots F (x) \dots))) = x + 1 $? Here $F$ is applied $F (x)$ times\\
\item (Putnam 1992, A1 modified) Find all $f: \mathbb{N} \to \mathbb{N}$ such that:\\
(a) $f(f(n)) = n$, for all integers $n$;\\
(b) $f(f(n + 2) + 2) = n$ for all integers $n$;\\
(c) $f(0) = 1$.
\item (Putnam 1971 B2) Let $F(x)$ be a real valued function defined for all real $x$ except for $x = 0$ and
$x = 1$ and satisfying the functional equation $F(x) + F(1-\frac{1}{x}) = 1 + x$. Find all functions $F(x)$ satisfying these conditions.
\item (a) (IMO 1987, 4)Prove that there is no function $f : \mathbb{N} \to \mathbb{N}$ which satisfies the functional equation $f(f(n)) = n + 1987$.\\
(b) Is there an $f : \mathbb{N} \to \mathbb{N}$ satisfying $f(f(n)) = n^2$?\\
(c) Is there a $g : \mathbb{R} \to \mathbb{R}$ such that $g(g(x)) = -x$? Is there a continuous $g$?
\item (IMO 1988, 3) A function $f$ defined on the positive integers (and taking positive integers values) is given by:
$\begin{matrix} f(1) = 1,\\ f(3) = 3 \\ f(2n) = f(n) \\ f(4n + 1) = 2f(2n + 1) - f(n) \\ f(4n + 3) = 3f(2n + 1) - 2f(n), \end{matrix}$
for all positive integers $n.$ Determine with proof the number of positive integers $\leq 1988$ for which $f(n) = n.$


\end{enumerate}
\end{xcb}
