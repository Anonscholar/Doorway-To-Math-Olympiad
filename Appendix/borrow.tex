\chapter{Appendix D: Borrowed Brillience}
As I have mentioned many times, this book is \cancel{steals} borrows brilliance from a lot of people who are frankly much more talented and accomplished than me. It is impotent that we pay tribute to them somewhere.
\section{Almost every part}
\begin{enumerate}
    \item \emph{Art of Problem Solving} website and forums: A lot of the theory, questions and their solutions were sourced from AOPS. If I even just mention the people whose work I directly used, this manuscript would get doubled. I am indebted beyond measure to the entire community.
    \item \emph{Brilliant.org} website: The website hosts a lot of free pages for different theorems and their exposition. I have taken a lot of inspiration from there along with questions. They could have easily kept these for the paid subscribers, but we can clearly see that they care more about math than money.
    \item \emph{Math Stack Exchange} forums: It is the reason this book was written in this lifetime. I am a very confused person, if the community there hadn't clarified the dumbest of my doubts, we wouldn't have this book. Also a lot of question and answers were sourced from here.
    \item \emph{Latex Stack Exchange} forums: It is the reason why this looks like a book and not a jumbled mess of words. It pains me that I can't feasibly mention everyone who helped by their username.
    \item \emph{Overleaf} latex editor: I didn't install Linux and then a latex editor and compiler and pdf viewer and what not to my computer. I used Overleaf as it was more organized, less cumbersome and easier to share and edit. All the die-hard Linux lovers are gonna lose their marbles. Guys, I don't know know how to do the installation and I will not learn until I absolutely need to.
    \item \emph{Math problem book} template by MAA: Used to give the book the 'book feel'.
    \item \emph{Random Hints} code by Evan Chan: The reason why we have the elegant random hints. It was the easiest one to use within overleaf.
\end{enumerate}
\section{Introductory Problems}
\begin{enumerate}
    \item \emph{Mathematical Circles: The Russian Experience} book by Dmitrii Vladimirovich Fomin, Ilia Itenberg, and S. Genkin: A classic book with a number of good questions. A must read for all primary school teachers.
    \item \emph{The USSR Olympiad problem book} Book by D. O. Shkliarskii: This book has a lot of interesting questions. Which require nothing more than the mind to solve. A clear reason why the Russian Olympiads are so fabled
    \item \emph{Friendship over Tea} video and problem by Arvind Gupta: I had decided while writing this book that I'll include Arvind Gupta sir at least once. He is such a legend. The way he has brought science and design education into the poorest parts of the world using toys from trash is commendable. Check him out, you'll not regret it. His Ted Talk was ranked as 2nd in the 5 favorite education talks.
    \item \emph{The Green Eyed Dragon and Other Mathematical Monsters} by David Morrin: A very fun book. I have long said to people that once a child turns of listening age, read a single problem from this every night and they'll develop the most wonderful sense of math and logic.
    \item {Counterfeit Coin Riddle} video by Jennifer Lu(Ted-Ed): Ted-Ed creates these beautiful animation videos explaining science concepts. They have entire series on world mythology, problems and basics of coding. In hostel, sometimes late at night, a bunch of me and my friends used to sit and see Ted-ed all night. We are all quite successful in math Olympiads, and ones watching something else all night, not quite...
    \item {The OTIS Excerpts} book by Evan Chen: Although it is borderline promotional material for his paid course, the book is quite educational.
\end{enumerate}
\section{Permutations and Combinations}
\begin{enumerate}
    \item \emph{Mastering AMC 10/12} by Sohil Rathee: A lot of the AMC 10 and 12 questions were sourced from the book. This is especially true for the part 1 of combinitorics. Sohil Rathi is an angel for going through the long history of AMC and painstakingly choosing the questions, and then putting it out for free.
    \item  \emph{Murderous Math: The Perfect Sausage and other fundamental formulas} book by Kjartan Poskitt: The muderous math series was some of first the non-academic math I had studied. A major influence on how I present math. 
    \item \emph{Introduction to Counting and Probability} handout by David Altzio: This handout in 2013-14 Math League gave me the idea and quite a few of the problems for The Guessing Game.
    \item  \emph{Permutation and Combination for IOQM 2023} lecture series by Abhay Mahajan(Vedantu Olympiad School): Some of the best lectures on combinitorics I have watched. No boubt would be much more popular if they were in English, unfortunately, they are in Hindi, so only fraction of you can enjoy them.
    \item \emph{Yale Putnam Handouts} by Pat Devlin: The long list handouts made by Pat for the Yale Putnam students was of enoumous use in sourcing some of the more difficult problems.
    \item \emph{Stanford Putnam Handouts} by Ravi Vakil: A legend of the math comunity, Ravi Vakil's handouts have been used all through the book, including in these for the 9 star problems.
\end{enumerate}
\section{Down The Rabbit Hole}
\begin{enumerate}
    \item \emph{Recursion in AIME} handout by Dylan Yu et al.: This handout, which is part of the Euclid's Orchid handouts, was a source for lot of the recursion theory and problems.
    \item \emph{Recurrence Relations} lecture by Prashant Jain: A great introductory video with a lot of classic examples. Sadly, its in Hindi, making it less accessible.
    \item \emph{Reccurence for INMO basics} lecture by Abhay Mahaajan: Abhay sir's question picking genius shines here. Every question is slightly harder and before you realize we are from AMC to IMO.
    \item \emph{IOQM 2022 practice sessions PnC} lecture by Prashant Jain: Prashant sir had introduced me to Catlan numbers here. The example over there is from Prashant sir's class.
    \item \emph{Bijections} by Yufei Zhao: The Catlan numbers were expanded upon using Yufai's handout.
    \item \emph{Counting in Two Ways} handout by Yufei Zhao: The idea of incidence matrix was completely taken from here.
    \item \emph{Introduction to Graph Theory} handout and powerpoint by Irene Lo: This 2019 Berkeley Math Circle handout was the basis for majority of the graph theory chapter.
    \item \emph{Olympiad Graph Theory} handout by Adam Kelly: This handout was mainly used for the questions in the graph theory chapter.
\end{enumerate}
\section{Algebra}
\begin{enumerate}
    \item \emph{Polynomials in AIME} handout by Dylan Lu et al.: Another great handout in the Euclid's orchid collection. Worth its weight in gold.
    \item \emph{Sequences and Series in the AMC and AIME} handout by Dylan Lu et al.: Also from the Euclid's orchid collection. Dylan and his gang have really great material on algebra.
    \item \emph{Series and Sequences} by David Altizio: Used mainly as a question source.
    \item \emph{A Brief Introduction to Olympiad Inequalities} handout by Evan Chen: This was the main reference for both of the inequalities chapters
    \item \emph{Inequalities} handout by Ananth Shyamal, Divya Shyamal, Kevin Yang, and Reece Yang: This was a handout for the Iowa City Math Circle. If any of the readers is in Iowa, I recommend visiting these peeps.
    \item \emph{Inequalities} handout by Dimitar Grantcharov: The handout made for Berkeley Math Circle while lacking in theory has a bunch of great questions.
\end{enumerate}
\section{The Red Pill}
\begin{enumerate}
    \item \emph{This Is the Calculus They Won't Teach You} video by A Well Rested Dog: This YouTube video made during the first Summer of Math Exposition talks about the history of calculus and was referenced as source if historical context.
    \item \emph{The Cartoon Guide to Calculus} book by Larry Gonick: This book was what I had used to study calculus and is my first recommendation to those learning it. A lot of the graphs and analogies were taken from this book.\\
    \item \emph{Limits and Continuity of Function} Lecture by Prashant Jain: This lecture, part of Bounce back series on Unacademy Atoms YouTube channel, was binged by me on a bus trip home. I have been able to solve some the most difficult questions of limits since then.
    \item \emph{Limits} lecture by Abhay Mahajan: These lectures(on the Vedantu JEE made EJEE channel) were the source of many of the questions\\
    \item \emph{Diffrentiation and Continuity} lecture by Abhay Mahajan: Again, used mainly for questions.
    \item \emph{Indefinite Integration} lecture by Abhay Mahajan: These lectures were the basis of most of the integration by substitution in the integration chapter. 
    \item \emph{DI method} videos by Steve Chow(Blackpen Redpen): The DI Method was first introduced to me by Steve Chow. His other videos on competition math, calculus and math for fun are just a delight to watch.
    \item \emph{Differential Equations} lecture notes by Nikenasih Binatari of State University of Yogyakarta: While the notes were for the semester course in differential equations, the first few chapters were directly referenced for the section.
    \item  \emph{Matrices and Determinents} lecture by Abhay Mahajan: These lectures were the basis for most of the linear algebra chapter.
    \item \emph{Lagrange Multipliers} videos by Khan Academy: The basis for the Lagrange multipliers section. There explanation was the simplest to understand and most straightforward, which allowed me to easily integrate it into the book.
    \item \emph{Lagrange Murderpliers Done Correctly} handout by Evan Chen: Most of the Lagrange inequalities were sourced from this handout.
    \item \emph{Sum Uses of Calculus} by David Altizio: The basis of the summation section. Parts of it were removed as they were too complex, but its overall a great handout.
\end{enumerate}
\section{Number Theory}
\begin{enumerate}
    \item \emph{Olypiad Number Thoery Through Challenging Problems} book by Justin Stevens: One of the most used pieces of reference for the entire number theory part. Great explanation, even better questions. Stevens is doing all of a favour by making the text available for free.
    \item \emph{Mordern Olympiad Number Theory} book by Aditya Khurmi: Another major reference for the Number Theory part. Covers some of the most complex number theory concepts in a digestible manner. Khurmi has given a gift by making this book free on AOPS. The book is so detailed and so beautiful that I had to literally fight myself for every complex concept I wanted to add.
    \item \emph{A Decade of the Berkeley Math Circle Volume I} book by Zvezdelina Stankova and Tom Rike: The session $4$ of this book was the inspiration for the name and the introductory passage for the first chapter of number theory. The general format of the number theory part is also inspired from here.
    \item \emph{Prime Numbers, Factors, and Division Tricks} handout by Linda Green: While more of her handouts occur in The Numbers Awaken, all of them have very creative examples and increase the question level gradually. Probably the best BMC handouts on number theory.
    \item \emph{Introduction to Modular Arithmetic} by David Altzio: It is so strange that David either gives me very complex questions or introductory questions. This one was had introductory questions.
    \item \emph{Bases Part I-II} problem set by Pratima Karpe: The source of almost all of the base questions.
    \item \emph{Putnam Problem Solving Seminar: Number Theory} handouts by Ravi Vakil: These handouts by Ravi Vakil from the years $2002, 2003, 2004, 2006$ was the source of the Putnam or Putnam adjacent problems.
    \item \emph{Functional equations} handout by Maxim Li: The only functional equation handout which I was able to understand in the first read. If it were not for this handout, I would myself never have understood functional equations in the first place.
    \item \emph{Functional Equations $\&$ Recurrence Relations} handout by Ted Alper: While I didn't use the Reccurence relations, or the analogy between functional equation and recurrence relations, I did use it for the purpose of getting introductory questions.
    \item \emph{Functional Equations} handout by Igor Ganichev: This was literally a list of problems. I solved all of them and used the ones I particularly found instructive.
    \item \emph{Putnam Problem Solving Seminar: Functional Equations} handouts by Vin De Silva, Mark Lucianovic and Ravi Vakil: This was also from the same seminars in the years $2005, 2006$. Again, they are source of the Putnam or Putnam adjacent problems.
    \item \emph{Introduction to Functional Equations} handout by Evan Chen: I really want to know want to know what Evan means by "Introduction" because this was by far the most complex reference. However, it had a lot of good examples and problems.
    \item \emph{Diophantine Equations} by David Altzio: This one is again somewhat on the easier side. Great for the introductory problems. However, I think David doesn't like NT much.
\end{enumerate}
\section{The Numbers Awaken}
\begin{enumerate}
\item \emph{Sledgehammers in number theory} by CJ Quines: The main reason the Bazooka chapter exists. Lovely handout!
\item \emph{Order's Modulo a Prime} by Evan Chen: While I complained about David's number thoery handouts being easy, I'll admit that Evan's handouts required more focus than boiling water with your minds.\\
\end{enumerate}