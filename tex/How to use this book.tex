\chapter*{How to use this book}
\section{Prerequisites}

Don't worry, I am not expecting you to perform calculus in your sleep or recite $\pi$ to a thousand decimal places.
But I do have some prerequisites, though, just to make sure you're in the right ballpark.\par
First off, you should be on friendly terms with basic arithmetic. Yep, that means adding, subtracting, multiplying, and dividing numbers
should be old hat for you. I won't judge if you occasionally rely on your trusty calculator, but let's aim for some proficiency here.
\par\medskip
Now, about those prime numbers and factoring and HCF and LCM. I'll dive into them later, 
but it'd be grand if you knew the basics. No sweat if you don't; I've got your back.\par\medskip
But wait, there's more! Beyond crunching numbers, you'll need to use a part of brain. 
A hefty dose of logical thinking will be used. I am talking creativity, the same thing you use in art and 
painting(yes, it's used in math as well).\par
Finally, You need to be ready to swap a bit of your overconfidence for some bona fide thinking prowess. 
If you are ready, then this book is your doorway to Math Olympiads.

\section{Structure of the book}
Every chapter has a few pages detailing the techniques and mathematical tools for that chapter.
The examples which are not followed by solutions are normally quite simple and meant to reinforce 
what you just read.\par
The examples followed by the answer are typically harder and meant to teach a problem-solving concept.\par
The problems at the end of the chapter start easy and move upward. 
They are meant to challenge you and get you out of the comfort zone while giving you the practice you require.
\par\medskip
The hints are numbered and appear in random order in \cref{app: hints}, and selected solutions 
in \cref{app: solns}. 
I have also tried to include the sources of the problems, so that a diligent reader can find more solutions online 
(for example on the Art of Problem Solving forums). A full listing of contest acronyms appears in \cref{app: borrow}.

\section{Approach}
Readers are encouraged to not be bureaucratic in their learning and move around as they see fit, e.g., 
skipping complicated sections and returning to them later, or moving quickly through familiar material.\par
However, I recommend not skipping problems out of anger and instead spending time before looking at the solution. 
However, I also don't recommend obsessing over a problem or being embarrassed of using the solutions or hints. \par

\section{Bluntly speaking}
\begin{mdframed}
    \textbf{Use Paper}
\end{mdframed}
See I care about the environment, but using paper while you read this book scratching notes
 and solving questions on paper rather than in your mind, is essential. \par
  A few papers won't kill the earth(and you can offset the pollution simply by planting a single tree which will give you 10000 sheets of 
 papers\footnote{Ribble Packaging Ltd. (2022). How much paper comes from one tree?}, You'll probably end up using less than that.)\par
You are not God. You cannot keep everything in your head(and if you think you can, you'll get proven wrong quite quickly), 
so please do yourself a favor and \textbf{use paper}.
