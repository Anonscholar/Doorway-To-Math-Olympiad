%    If you need symbols beyond the basic set, uncomment this command.
\usepackage{tikz}
\usepackage{amssymb,amsthm,amssymb}
\usepackage{float}
\usepackage[]{tcolorbox}
\usepackage{mdframed}
\usepackage{graphicx}
\usepackage{float}
\usepackage{booktabs}
\usepackage{epigraph}
\usepackage{xargs}
\usepackage{centernot}
\usepackage{mathtools}

\usepackage[linktoc=page]{hyperref}
\hypersetup{
colorlinks=true,
linkcolor=green!70!black,
citecolor=green,
}
\usepackage{cleveref}


\usepackage{natbib}
\usepackage{forloop}
\usepackage{pgfmath}
\usepackage{etoolbox}
\usepackage{ifthen}
\usepackage{answers}
\usepackage{cancel}
\usepackage{mdframed}
%\usepackage{draftwatermark}
% Watermark
%\SetWatermarkLightness{ 0.9 }
%\SetWatermarkText{DRAFT}
%\SetWatermarkScale{ 3 }
%Expected Value
\newcommand{\E}[1]{\mathbb{E}({#1})}
%LCM command
\DeclareMathOperator*{\lcm}{lcm}
%true and false
\DeclareMathOperator{\true}{true}
\DeclareMathOperator{\false}{false}
% Points commad
\newcommand{\points}[1]{[$#1 \star$]}
% Setup counters
\newcounter{hindex}\setcounter{hindex}{0}
\newcounter{hintcounter}\setcounter{hintcounter}{0}
% Define \addhint and \gethint
\newcommand\addhint[1]{%
    \stepcounter{hintcounter}%
    \ref{hint:\thehintcounter}%
    \expandafter\gdef\csname hintlist\thehintcounter\endcsname{#1}%
}
\newcommand\gethint[1]{%
    \item \csname hintlist#1\endcsname \label{hint:#1}
}
\newenvironment{hint}{\footnotesize \normalfont \textbf{Hints}:}{\hspace{-0.5ex}}
\pgfmathsetseed{65536} % or any other number: sets the random seed

\newtheorem{theorem}{Theorem}[chapter]
\newtheorem{lemma}[theorem]{Lemma}

\theoremstyle{definition}
\newtheorem{definition}[theorem]{Definition}
\newtheorem{example}[theorem]{Example}
\newtheorem{xca}[theorem]{Exercise}

\theoremstyle{remark}
\newtheorem{remark}[theorem]{Remark}

\numberwithin{section}{chapter}
\numberwithin{equation}{chapter}
\setcounter{chapter}{0}
