%    If you need symbols beyond the basic set, uncomment this command.
\usepackage{tikz}
\usepackage{amssymb, amssymb}
\usepackage{float}
\usepackage{lua-check-hyphen}
\usepackage[dvipsnames]{xcolor}
\usepackage{amsthm}
\usepackage{thmtools,xpatch}
\usepackage[framemethod=TikZ]{mdframed}
\raggedbottom
\usetikzlibrary{shadows}
% https://tex.stackexchange.com/a/292090/76888
% https://github.com/marcodaniel/mdframed/issues/12

\usepackage{graphicx}
\usepackage{float}
\usepackage{booktabs}
\usepackage{epigraph}
\usepackage{xargs}
\usepackage{centernot}
\usepackage{siunitx}
\usepackage{mathtools}


\usepackage[linktoc=page]{hyperref}
\hypersetup{
colorlinks=true,
linkcolor=TealBlue!70!black,
citecolor=TealBlue,
}
\usepackage{cleveref}

\usepackage{natbib}
\usepackage{forloop}
\usepackage{pgfmath}
\usepackage{etoolbox}
\usepackage{ifthen}
\usepackage{answers}
\usepackage{cancel}

\usepackage{todonotes}

\xpatchcmd{\endmdframed}
{\aftergroup\endmdf@trivlist\color@endgroup}
{\endmdf@trivlist\color@endgroup\@doendpe}
{}{}
%\usepackage{draftwatermark}
% Watermark
%\SetWatermarkLightness{ 0.9 }
%\SetWatermarkText{DRAFT}
%\SetWatermarkScale{ 3 }
%Expected Value
\newcommand{\E}[1]{\mathbb{E}({#1})}
%LCM command
\DeclareMathOperator*{\lcm}{lcm}
%true and false
\DeclareMathOperator{\true}{true}
\DeclareMathOperator{\false}{false}
% Points commad
\newcommand{\points}[1]{[$#1 \star$]}
% Setup counters
\newcounter{hindex}\setcounter{hindex}{0}
\newcounter{hintcounter}\setcounter{hintcounter}{0}
% Define \addhint and \gethint
\newcommand\addhint[1]{%
    \stepcounter{hintcounter}%
    \ref{hint:\thehintcounter}%
    \expandafter\gdef\csname hintlist\thehintcounter\endcsname{#1}%
}
\newcommand\gethint[1]{%
    \item \csname hintlist#1\endcsname \label{hint:#1}
}
\newenvironment{hint}{\footnotesize \normalfont \textbf{Hints}:}{\hspace{-0.5ex}}
\pgfmathsetseed{65536} % or any other number: sets the random seed



\mdfdefinestyle{blackbox}{%
    frametitlerulewidth=1pt,
    frametitlerule=true,
    roundcorner=5pt,
    linewidth=1pt,
    skipabove=15pt,
    skipbelow=2pt,
    frametitlefont=\bfseries,
    innertopmargin=12pt,
    innerbottommargin=8pt,
    nobreak=true,
    backgroundcolor=white,
    linecolor=black,
}

\declaretheoremstyle[
mdframed={style=blackbox},
headpunct={},
postheadspace=\newline,
postheadhook={\rule[.6ex]{\linewidth}{0.4pt}\\},
]{thmbox}

\declaretheoremstyle[
headfont=\normalfont\bfseries,
spaceabove=25pt,
spacebelow=25pt,
bodyfont=\normalfont
]{basehead}

\declaretheorem[style=basehead,name=Theorem,numberwithin=section]{theorem}
\declaretheorem[style=basehead,name=Lemma,sibling=theorem]{lemma}
\mdtheorem[style=blackbox]{example}[subsection]{Example}

\declaretheorem[style=basehead,name=Definition,sibling=theorem]{definition}
\declaretheorem[style=basehead,name=Definition,sibling=theorem]{remark}

\numberwithin{section}{chapter}
\numberwithin{equation}{chapter}
\setcounter{chapter}{0}
%%macros

\newcommand{\CC}{\mathbb C}
\newcommand{\FF}{\mathbb F}
\newcommand{\NN}{\mathbb N}
\newcommand{\NNO}{\mathbb N_{\ge 0}}
\newcommand{\ZZO}{\mathbb Z_{\ge 0}}
\newcommand{\QQ}{\mathbb Q}
\newcommand{\RR}{\mathbb R}
\newcommand{\ZZ}{\mathbb Z}